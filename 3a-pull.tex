\small
\underline{\uppercase{the pull}}
\begin{itemize}
    \item [7.3] After signalling readiness all offensive players must stand with one foot on their defending goal line without changing location relative to one another until the pull is released.
    \item [7.4] After signalling readiness all defensive players must keep their feet entirely behind the vertical plane of the goal line until the pull is released.
    \item [7.5] If a team breaches 7.3 or 7.4 the opposing team may call a violation (“offside”). This must be called before the offence touches the disc (7.8 still applies).
        \begin{itemize}
            \item[7.5.1] If the defence chooses to call offside, the thrower must establish a pivot point as per 7.9, 7.10, 7.11, or 7.12 and then play restarts as soon as possible as if a time-out had been called at that location.
            \item[7.5.2] If the offence chooses to call offside, they must let the disc hit the ground untouched and then resume play as if a brick has been called (no check is required).
        \end{itemize}
    \item[7.8] If an offensive player, in-bounds or out-of-bounds, touches the disc before it hits the ground, and the offensive team fails to subsequently establish possession, that is a turnover (a “dropped pull”).
    \item[7.9] If an offensive player catches the pull and subsequently establishes possession, they must establish a pivot point at the location on the playing field nearest to where possession is established, even if that pivot point is in their defending end zone.
    \item[7.11] If the disc initially contacts the playing field and then becomes out-of-bounds without contacting an offensive player, the thrower must establish a pivot point where the disc first crossed the perimeter line, or the nearest location in the central zone if that pivot point would be in their defending end zone
    \item[7.12] If the disc contacts the out-of-bounds area without first touching the playing field or an offensive player, the thrower may establish a pivot point either at the brick mark closest to their defending end zone, or at the location on the central zone closest to where the disc went out-of-bounds (Section 11.8). The binding brick option must be signalled before the disc is picked up, by any offensive player fully extending one arm overhead and calling “brick”.
    \item[8.4] Any player may attempt to stop a disc from rolling or sliding after it has hit the ground.
\end{itemize}
\begin{center}[a]\end{center}
