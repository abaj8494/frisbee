\small{
    \begin{itemize}
        \item first to 15 wins
        \item gamelength = 100 minutes
        \item you can play with a minimum of 5 players
        \item half time at 8 points
        \item two timeouts per half. 75 seconds each$^\dagger$
    \end{itemize}
}

20.2. After the start of a point and before both teams have signalled readiness, a player from either team may
call a time-out. The time-out extends the time between the start of the point and subsequent pull by
seventy-five (75) seconds.
20.3. After the pull only a thrower with possession of the disc may call a time-out. The time-out starts when
the “T” is formed, and lasts seventy-five (75) seconds. After such a time-out:
20.3.1. Substitutions are not allowed, except for injury.
20.3.2. Play is restarted at the pivot location.
20.3.3. The thrower must remain the same.
20.3.4. All other offensive players must establish a stationary position, at any location.
20.3.5. Once the offensive players have selected positions, defensive players must then establish a
stationary position, at any location.
20.3.6. The stall count restarts at maximum nine (9). However if the marker has been switched, the
stall count restarts at “Stalling one (1)”.
20.4. If the thrower attempts to call a time-out while play is live and when their team has no remaining timeouts, play is stopped. The marker must add two (2) seconds to the stall count they would have
restarted play on before restarting play with a check. If this results in a stall count of ten (10) or above,
this is a "stall-out" turnover.

\begin{center}[1]\end{center}
