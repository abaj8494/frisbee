\underline{\uppercase{Misc}}

\begin{itemize}
    \item 13.10. If the turnover location is in the offence’s attacking end zone, the thrower must establish a pivot point at the nearest location on the goal line.
    \item 13.11. If the turnover location is in the offence’s defending end zone, the thrower may choose where to establish a pivot point:
    \item 13.11.1. at the turnover location, by staying at the turnover location or faking a pass; or
    \item 13.11.2. at the nearest location on the goal line to the turnover location, by moving from the turnover location.
    \item 13.11.2.1. The intended thrower, before picking up the disc, may signal the goal line option by fully extending one arm above their head.
    \item 13.11.3. Immediate movement, staying at the turnover location, faking a pass, or signaling the goal line option, determines where to establish a pivot point and cannot be reversed.
\end{itemize}


\subsection*{Misc}

\begin{itemize}
    \item 1.11. Players and captains are solely responsible for making and resolving all calls.
    \item 11.2. The out-of-bounds area consists of the ground which is not in-bounds and everything in contact with it, except for defensive players, who are always considered “in-bounds”.
    \item 12.3. If offensive and defensive players catch the disc simultaneously, the offence retains possession.
\end{itemize}

\begin{center}[c]\end{center}
